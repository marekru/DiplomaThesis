\chapter*{Úvod}
\addcontentsline{toc}{chapter}{Úvod} 	
	
Vizualizácia informácií a vizuálna analýza dát sú dnes veľmi vyvíjaným a moderným odvetvím počítačovej grafiky. Aplikácie vizualizácie informácií sú rôzne od vedeckého výskumu cez finančné analýzy až po komerčné použitie vo verejných médiách. Jej hlavnou úlohou je v prvom rade využitie ľudských kognitívnych schopností pre lepšie porozumenie dát, teda z nezrozumiteľného zhluku dát vytvoriť zrozumiteľnú vizuálnu reprezentáciu.

Svoje využitie našla vizualizácia informácií aj v procese verifikácie predpovedných modelov počasia. Dnešné metódy verifikácie predpovedí sa sústreďujú na numerický popis výkonu jednotlivých predpovedných modelov pomocou rôznorodých štatistických metód. Vo výsledku sa teda dáta opisujú ďalšími dátami, avšak práve na vizualizáciu týchto dát sa nekladie príliš veľký dôraz.

V našej práci sme preštudovali používané štatistické metódy pre verifikáciu spojitých premenných, ktoré predpovedá model \textit{WRF}. Hlavným cieľom práce sa však stal návrh a implementácia vizualizačných techník špeciálne navrhnutých pre potreby verifikácie. Sústredili sme sa obzvlášť na to, aby sme ušetrili cenný vizuálny priestor bez straty potrebných informácií. Návrh vizualizácie vychádzal z bežnej meteorologickej praxe, a to zo zaužívaných meteorologických pojmov, štatistických modelov, vizualizačných techník a taktiež úloh, ktoré užívateľ pri verifikácii vykonáva.
