\chapter{Úvod}

Kocka sa stala hlavným symbolom tejto práce, a tak sme nami navrhnutú aplikáciu nazvali \textit{Cubo}, čo vlastne aj znamená po španielsky kocka. Tento názov sa nápadne podobá aj na anglické slovo \textit{cube}, od ktorého sme sa pôvodne odrazili. Takýmto spôsobom chceme výrazne naznačiť, čomu sa program venuje.

Táto kapitola v stručnosti popisuje existujúce skutočnosti, ktoré nás podnietili k vytvoreniu tejto aplikácie a následne stanovujeme ciele, ktoré chceme v práci dosiahnuť.

\section{Motivácia}

Objemová grafika, alebo inak tiež voxelová grafika, je pomerne mladá oblasť počítačovej grafiky a len nedávno našla svoje uplatnenie aj v komerčnej sfére, čo je prvý dôležitý krok k jej skutočnému rozvoju. Jej ďalšia budúcnosť stojí a padá na hardvérovej podpore a dúfame, že je len otázkou času, kedy vzniknú prostriedky umožňujúce vývojárom spracovávanie veľkého množstva voxelov v reálnom čase.

Ako som už spomenul vyššie, voxelová grafika našla svoje uplatnenie v komerčnej sfére ako napríklad v herných enginoch, animácii alebo tvorbe lego skulptúr a podobne. Jej ohromnou výhodou v hrách je, že sa objekty skladajú z veľkého množstva častíc a navyše uchovávajú v sebe objem, čo sa dá využiť napríklad pri reálnejšej simulácii fyziky alebo úplnej deštrukcii herného prostredia. Poznanie, že voxelová grafika sa objavuje v hrách priamo implikuje potrebu nástroja na vytváranie herných modelov. Voxelový editor môže mať tiež iné využitie a to na tvorbu zaujímavých 3D kockatých renderov, ktoré sú v dnešnej dobe veľmi populárne, a tak sa často objavujú na rôznych plagátoch alebo internetových reklamách.

Krása kociek uchvátila mnohých a nie jeden baží po vytvorení si svôjho vlastného voxelového modelu. Rozhodli sme sa teda o zhotovenie praktického kresliaceho nástroja v 3D, uplatňujúc skúsenosti získané pri práci v 2D grafických programoch. 

\section{Cieľ}
Našim hlavným cieľom je vytvorenie editačného nástroja na úpravu voxelových objektov a scén. Tento cieľ sa skladá z viacerých podcieľov, ktoré sme si stanovili.

Jedným z cieľov práce je uviesť čitateľa do problematiky zadefinovaním základných pojmov objemovej grafiky, akými sú napríklad \textit{voxel}, \textit{objemové dáta} a podobne. V rámci tohoto cieľa chceme čo najstručnejšie opísať výhody a nevýhody voxelovej grafiky, jej dôležité postavenie v rámci počítačovej grafiky a taktiež jej využitie mimo vedeckej sféry.

Ďalším cieľom je opísať existujúce riešenia, ich silné a slabé stránky a ich vplyv na našu prácu. Následne chceme tieto skúsenosti aplikovať v implementačnej fáze, kde chceme podrobne opísať jednotlivé časti programu, ich možnosti a taktiež metódy, ktoré sme použili pri ich vytváraní. 

Na záver chceme zhrnúť výsledky jednotlivých častí programu, ktoré sme v práci dosiahli a taktiež možnosti ďalšieho vývoja aplikácie v budúcnosti.
