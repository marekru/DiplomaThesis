\chapter{Výsledky a Záver}

\section{Testovanie}

\section{Demonštrácia}

\section{Záver}
v práci sme uviedli problém verifikácie predpovedných modelov počasia so zameraním na verifikáciu predpovede spojitej premennej v jednom bode. Zhrnuli sme používané štatistické metódy na meranie chyby predpovede a navrhli sme spôsob všeobecného výpočtu kumulovanej chyby, ktorý sme v závere využili aj pri implementácii.

Taktiež sme preskúmali doterajšie riešenia jednak z pohľadu verifikačného softvéru, ale aj z pohľadu vizualizačných techník vo verifikácii. Uviedli sme slabé a silné stránky jednotlivých softvérových riešení a zhrnuli sme ich v prehľadnej tabuľke. Taktiež sme opísali a analyzovali vizualizácie používané vo verifikácii a načrtli ich zameranie a spôsob použitia.

Ďalej sme charakterizovali vstupné dáta a špecifikovali požiadavky používateľov, na základe ktorých sme navrhli rôzne spôsoby vizualizácie, ktoré sme implementovali ako doplnok JavaScriptovej knižnice D3. V závere práce sme následne vykonali porovnávacie testovanie našej vizualizácie s pôvodným spôsobom používaným meteorológmi.

Výsledkom práce je teda verifikačný nástroj schopný spracovávať dáta z rôznych zdrojov, ponúkajúc funkcionalitu pre verifikáciu predpovedí spojitej premennej s použitím špeciálne navrhnutej vizualizácie pre tieto účely.

Za hlavný prínos tejto práce považujeme práve návrh špecializovanej vizualizácie, ktorá síce nevyniká originalitou, ale pri jej vzniku sme navrhli mnoho drobných vylepšení smerujúcim k lepšiemu, rýchlejšiemu a jednoduchšiemu porozumeniu dát. Príkladom môže byť



