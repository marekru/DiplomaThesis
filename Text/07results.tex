\chapter{Výsledky a Záver}

\section{Testovanie}

\section{Demonštrácia}

\section{Záver}
v práci sme uviedli problém verifikácie predpovedných modelov počasia so zameraním na verifikáciu predpovede spojitej premennej v jednom bode. Zhrnuli sme používané štatistické metódy na meranie chyby predpovede a navrhli sme spôsob všeobecného výpočtu kumulovanej chyby, ktorý sme v závere aj implementovali.

Taktiež sme preskúmali doterajšie riešenia jednak z pohľadu verifikačného softvéru, ale aj z pohľadu vizualizačných techník vo verifikácii. Uviedli sme slabé a silné stránky jednotlivých softvérových riešení a zhrnuli sme ich v prehľadnej tabuľke. Taktiež sme analyzovali vizualizácie používané vo verifikácii.

Ďalej sme charakterizovali vstupné dáta a špecifikovali požiadavky používateľov, na základe ktorých sme navrhli rôzne spôsoby vizualizácie, ktoré sme implementovali ako doplnok JavaScriptovej knižnice D3.

Výsledkom práce je teda verifikačný nástroj schopný spracovávať rôznorodé dáta, vypočítať verifikačné štatistiky a zrozumiteľne vizualizovať výsledné dáta.

Hlavným prínosmi práce:
- navrh specializovanej vizualizácie zameranej na verifikaciu \\
- navrh pruhoveho diagramu \\


 