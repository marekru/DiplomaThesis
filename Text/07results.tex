\chapter{Výsledky a Záver}

\section{Testovanie}
Pri testovaní vizualizácie sme postupovali podľa toho a toho %TODO cite.

\subsection{Testovacia procedúra}
Testovanie prebiehalo prostredníctvom niekoľkých úloh, ktoré sme navrhli na základe špecifikácie požiadaviek na vizualizáciu uvedených v sekcii \ref{sec:spec}. 

Každému subjektu bola predstavená základná funkcionalita systému v krátkom 15 minútovom návode. Následne subjekt obdržal testovací formulár (pozri prílohu \ref{sec:testform}) obsahujúci 7 rôznych úloh, ktoré mal subjekt vykonať a po vykonaní každej z nich do formulára zapísať výsledok svojho skúmania.

Pri testovaní sme jednak vyhodnocovali správnosť odpovedí ale aj rýchlosť vykonania jednotlivých úloh. Taktiež sme dokumentovali interakciu s vizualizáciou počas vykonávania úlohy pomocou \textit{screen capture}.

\subsection{Výsledky testovania}

\begin{table}[h]
\centering
\caption{Výsledky testovania}
\label{table:results}
\begin{tabular}{|l|l|l|l|}
\hline
\textbf{Úloha} & \textbf{Počet správnych} & \textbf{Počet nesprávnych} & \textbf{Priemerný čas} \\ \hline
1              &                          &                            &                                    \\ \hline
2              &                          &                            &                                    \\ \hline
3              &                          &                            &                                    \\ \hline
4              &                          &                            &                                    \\ \hline
5              &                          &                            &                                    \\ \hline
6              &                          &                            &                                    \\ \hline
7              &                          &                            &                                    \\ \hline
8              &                          &                            &                                    \\ \hline
\textbf{Suma}  &                          &                            &                                    \\ \hline
\end{tabular}
\end{table}

%TODO \section{Demonštrácia}

\section{Záver}
v práci sme uviedli problém verifikácie predpovedných modelov počasia so zameraním na verifikáciu predpovede spojitej premennej v jednom bode. Zhrnuli sme používané štatistické metódy na meranie chyby predpovede a navrhli sme spôsob všeobecného výpočtu kumulovanej chyby, ktorý sme v závere využili aj pri implementácii.

Taktiež sme preskúmali doterajšie riešenia jednak z pohľadu verifikačného softvéru, ale aj z pohľadu vizualizačných techník vo verifikácii. Uviedli sme slabé a silné stránky jednotlivých softvérových riešení a zhrnuli sme ich v prehľadnej tabuľke. Taktiež sme opísali a analyzovali vizualizácie používané vo verifikácii a načrtli ich zameranie a spôsob použitia.

Ďalej sme charakterizovali vstupné dáta a špecifikovali požiadavky používateľov, na základe ktorých sme navrhli rôzne spôsoby vizualizácie, ktoré sme implementovali ako doplnok JavaScriptovej knižnice D3. V závere práce sme následne vykonali porovnávacie testovanie našej vizualizácie s pôvodným spôsobom používaným meteorológmi.

Výsledkom práce je teda verifikačný nástroj schopný spracovávať dáta z rôznych zdrojov, ponúkajúc funkcionalitu pre verifikáciu predpovedí spojitej premennej s použitím špeciálne navrhnutej vizualizácie pre tieto účely. 

\subsubsection{Prínos}

Za hlavný prínos tejto práce považujeme práve návrh špecializovanej vizualizácie, ktorá síce nevyniká obrovskou inováciou, ale pri jej vzniku sme navrhli mnoho drobných originálnych vylepšení smerujúcim k lepšiemu, rýchlejšiemu a jednoduchšiemu porozumeniu dát. Príkladom môže byť návrh prehľadovej vizualizácie verifikačných dát, návrh riešenia viacerých škál pri \mbox{mnoho-čiarovom} diagrame, zovšeobecnenie re-dizajnu krabicového diagramu od \mbox{Bada et. al.} \cite{Bade} na takzvaný \textit{Pruhový diagram}, využitie dvojtónového pseudofarbenia pri kompaktnej vizualizácii distribúcie dát pomocou grafu hustoty a podobne.

\subsubsection{Budúca práca}

Na výskum a vývoj neustále existuje priestor a v našej práci ho vidíme určite mnoho obzvlášť v štyroch oblastiach:
\begin{enumerate}
	\item \textit{Získavanie dát} - V práci sme implementovali 3 typy zdrojov: Grib, CSV a Web. Avšak existuje mnoho ďalších zdrojov dát, z kade možno získať pozorovania, rovnako ako aj predpovede. Príkladom je napríklad \textit{databáza}, \textit{XML}, alebo \textit{XLS} súbory, takzvané \textit{bloky} systému IMS4, kde sú aj merania a mnohé iné ďalšie zdroje. 
	\item \textit{Konfigurácia systému a vizualizácie} - Konfigurovanie systému v momentálnom stave nie je veľmi user-friendly úloha. V budúcej práci očakávame návrh riešenia, ktoré skryje pred užívateľom trocha neprehľadné nastavenia systému.
	\item \textit{Interaktívna manipulácia s dátami} - Momentálne riešenie neponúka veľmi dôležitú súčasť moderných vizualizačných nástrojov a tým je manipulácia s dátami priamo na obrazovke. Príkladmi operácií môže byť filtrovanie dát na základe rôznych kľúčov alebo selekcia vybranej vzorky dát pomocou \textit{brushingu} a podobne.
	\item \textit{Porovnávanie modelov a staníc} - Pri návrhu vizualizácie sme nezvažovali možnosť porovnávania výkonu viacerých predpovedných modelov alebo porovnávanie predpovedí pre rôzne stanice súčasne. Riešenie takéhoto problému by pravdepodobne vyžadovalo zásadný re-dizajn vizualizácie.
\end{enumerate}