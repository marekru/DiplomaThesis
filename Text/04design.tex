\chapter{Návrh riešenia}


\section{Návrh systému}

\section{Návrh vizualizácie}

\subsection{Charakteristika dát}

\subsection{Špecifikácia požiadaviek na vizualizáciu}

\subsection{Návrh farebnej palety}


\subsection{Návrh vizualizácie distribúcie chýb}
Pri verifikácii predpovede spojitej premennej sme použili štatistické metódy spomenuté v sekcii \ref{sec:errormeasurement}, ktorých výsledok sme následne vizualizovali. Pôvodné dáta však zostali skryté za použitým matematickým modelom, a tak sme stratili informáciu o distribúcii chyby. Pri verifikácii sa štandardne používajú dve metódy na priamu, či nepriamu vizualizáciu a analýzu distribúcie, ktoré sme opísali v sekcii \ref{sec:prevvis}. Týmito metódami sú bodový graf (pozri podsekciu \ref{subsec:scatterplot}) a krabicový diagram (pozri podsekciu \ref{subsec:boxplot}).

Pri návrh vizualizácie sme vyskúšali niekoľko vizualizačných techník, .... bla bla bla %TODO something logical here

\subsubsection{Density plot}


\subsubsection{Kvantilový diagram} %TODO toto je v skutocnosti daco insie, musim nastuduvat


\subsubsection{Funkcionálny krabicový diagram}

