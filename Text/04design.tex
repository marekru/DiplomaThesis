\chapter{Návrh riešenia}


\section{Návrh systému}

\section{Návrh vizualizácie}

\subsection{Charakteristika dát}

\subsection{Špecifikácia požiadaviek na vizualizáciu}

\subsection{Návrh farebnej palety}


\subsection{Návrh vizualizácie distribúcie chýb}
Pri verifikácii predpovede spojitej premennej sme použili štatistické metódy spomenuté v sekcii \ref{sec:errormeasurement}, ktorých výsledok sme následne vizualizovali. Pôvodné dáta však zostali skryté za použitým matematickým modelom, a tak sme stratili informáciu o distribúcii chyby. Pri verifikácii sa štandardne používajú dve metódy na priamu, či nepriamu vizualizáciu a analýzu distribúcie, ktoré sme opísali v sekcii \ref{sec:prevvis}. Týmito metódami sú bodový graf (pozri podsekciu \ref{subsec:scatterplot}) a krabicový diagram (pozri podsekciu \ref{subsec:boxplot}).

Pri návrh vizualizácie sme vyskúšali niekoľko vizualizačných techník a zvážili ich silné a slabé stránky. 

\subsubsection{Graf hustoty} %Density plot


\subsubsection{Kvantilový diagram} 


\subsubsection{Funkčný krabicový diagram}
Pre pochopenie dát je dôležité, aby sme sa vedeli pozrieť na hodnoty v ich kontexte. Všetky predošlé techniky uvažovali o chybe ako o samostatnej hodnote pre určitý predpovedný čas, avšak chyby sa nenachádzajú len v kontexte predpovedného času, ale aj v kontexte konkrétnej predpovede. Preto môžme uvažovať o predpovediach ako o funkciách $ x_{i}(t) $, kde $ i \in \{1..n\}$ je poradie predpovede a $ t \in I $ je čas predpovede, kde $ I $ je časový interval predpovede z $ \mathbb{R} $ (v našom prípade sa jednalo o dvojdňovú, teda 48 hodinovú predpoveď).

Takýmto spôsobom sme sa dostali do novej situácie, kedy nechceme vizualizovať distribúciu jednotlivých chýb, ale celých predpovedí, ktoré chápeme ako funkcie. Na riešenie tohto problému existuje niekoľko spôsobov, z ktorých sme si zvolili \textit{funkčný krabicový diagram} \cite{FunctionalBoxplot}, keďže myšlienkovo vychádza z klasického krabicového diagramu, ktorý je jednak na túto situáciu vhodný ale je tiež medzi užívateľmi dobre známy a zaužívaný.

Ako sme spomenuli v časti \ref{fig:boxplot}, klasický krabicový diagram potrebuje na svoju konštrukciu 5 hodnôt: 3 kvartily a 2 extrémy. Aby sme tieto hodnoty našli pre funkcie, musíme ich vedieť porovnať a povedať, ktorá je "väčšia" alebo "menšia". Autori funkčného krabicového diagramu riešia problém s využitím takzvanej hĺbky pásma (\textit{band depth}) \cite{BandDepth}. Grafom $ G $ funkcie $ x $ je množina bodov $ G = \{ (t,x(t)) : t \in I \} $. Pásmo $ \mathcal{B} $ (\textit{band}) v $ \mathbb{R}^{2}  $ ohraničené krivkami $ x_{i_{1}}, x_{i_{2}}, .. , x_{i_{k}} $, kde $ k \geq 2 $ je definované takto:
\[
	\mathcal{B}(x_{i_{1}}, x_{i_{2}}, .. , x_{i_{k}}) = \{ (t,y) : t \in I, \min_{r=1..k}x_{i_{r}}(t) \leq y \leq \min_{r=1..k}x_{i_{r}}(t) \}
\]
Pásmo $ \mathcal{B} $ je teda množina všetkých bodov existujúcich medzi extrémami všetkých kriviek, ktoré doň vstupujú ako parameter. 
Pomocou týchto dvoch funkcií môžme definovať pomocnú funkciu $ BD_{n}^{(j)}(x) $ pre krivku $ x $, ktorá vyzerá takto:
\[
	BD^{(j)}_{n}(x) = {n \choose j}^{-1} \sum_{1 \leq i_{1} < i_{2} < ... < i_{j} \leq n} \mathcal{I}\{ G(x) \subset \mathcal{B}(x_{i_{1}}, ... ,x_{i_{j}}) \}, j \geq 2
\]
kde $ j $ je počet kriviek definujúce pásmo $ \mathcal{B} $, $ n $ je celkový počet kriviek a $ \mathcal{I} $ je takáto funkcia:
\[
	\mathcal{I}(x) = \left\{
	\begin{array}{ll}
	1 & \mbox{ak platí x}  \\
	0 & \mbox{ak neplatí x} 
	\end{array}
	\right.
\]
Pomocná funkcia $ BD $ pre krivku $ x $ definuje pomer všetkých $ j $-tic, v ktorých pásme sa graf $ G(x) $ nachádza ku všetkým $ j $-ticiam, ktoré možno z $ n $ kriviek vybrať.

\subsubsection{Porovnanie metód ...}

Tu bude pekný obrázok a obkeci.