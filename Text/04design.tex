\chapter{Návrh riešenia}


\section{Návrh systému}

\section{Návrh vizualizácie}

\subsection{Charakteristika dát}

\subsection{Špecifikácia požiadaviek na vizualizáciu}

\subsection{Návrh farebnej palety}


\subsection{Návrh vizualizácie distribúcie chýb}
Pri verifikácii predpovede spojitej premennej sme použili štatistické metódy spomenuté v sekcii \ref{sec:errormeasurement}, ktorých výsledok sme následne vizualizovali. Pôvodné dáta však zostali skryté za použitým matematickým modelom, a tak sme stratili informáciu o distribúcii chyby. Pri verifikácii sa štandardne používajú dve metódy na priamu, či nepriamu vizualizáciu a analýzu distribúcie, ktoré sme opísali v sekcii \ref{sec:prevvis}. Týmito metódami sú bodový graf (pozri podsekciu \ref{subsec:scatterplot}) a krabicový diagram (pozri podsekciu \ref{subsec:boxplot}).

Pri návrh vizualizácie sme vyskúšali niekoľko vizualizačných techník a zvážili ich silné a slabé stránky. 

\subsubsection{Graf hustoty} %Density plot


\subsubsection{Kvantilový diagram} 


\subsubsection{Funkcionálny krabicový diagram}
Pre pochopenie dát je dôležité, aby sme sa vedeli pozrieť na hodnoty v ich kontexte. Všetky predošlé techniky uvažovali o chybe ako o samostatnej hodnote pre určitý predpovedný čas, avšak chyby sa nenachádzajú len v kontexte predpovedného času, ale aj v kontexte konkrétnej predpovede. Preto môžme uvažovať o predpovediach ako o funkciách $ x_{i}(t) $, kde $ i \in \{1..n\}$ je poradie predpovede a $ t \in I $ je čas predpovede, kde $ I $ je časový interval predpovede z $ \mathbb{R} $ (v našom prípade sa jednalo o dvojdňovú, teda 48 hodinovú predpoveď).


	


\subsubsection{Porovnanie metód ...}

Tu bude pekný obrázok a obkeci.