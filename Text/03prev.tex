\chapter{Predchádzajúce riešenia}

\section{Verifikačný softvér}

Verifikácia predpovedných modelov počasia je úloha dokonale stvorená pre automatizáciu. 
Z tohto dôvodu meteorológovia začali využívať dostupný štatistický softvér 
a neskôr boli taktiež vyvíjané špecializované nástroje určené pre verifikáciu.
Môžeme teda rozdeliť verifikačný softvér do dvoch základných kategórií a to \textit{štatistický} a \textit{špecializovaný}, ktorý je zväčša podporovaný rôznymi národnými a medzinárodnými organizáciami.

\subsection{Štatistický softvér}

\subsubsection{Tabuľkový softvér}
Napriek tomu, že je tabuľkový softvér na výpočet štatistík zamietnutý komunitou vedcov a štatistikov ako nevhodný a neprofesionálny, tak je využívaný, a to pomerne často, aj medzi vedeckými kruhmi. 
Výhodou je, že novému užívateľovi umožňuje okamžite vidieť všetky kroky v základných procedúrach verifikácie. [FVS.pdf] %TODO
Najznámejší kus softvéru z pomedzi komerčných produktov je \textit{Microsoft Excel} a z voľne dostupných je jeho opensoruce náprotivok \textit{Open Office Calculate}. Oba programy zahrňujú základné štatistické funkcie ako napríklad \textit{MAE} pre spojité predpovede(vid kapitola 2) a taktiež umožňujú generovanie jednoduchých grafov na základe tabuľkových dát.



\subsubsection{MATLAB}

\subsubsection{Minitab}

\subsubsection{R}

\subsubsection[SAS]{Statistical Analysis Software (SAS)}

\subsubsection[IDL]{Interactive Data Language (IDL)}

\subsection{Špecializovaný softvér}

\subsubsection[NCL]{NCAR Command Language (NCL)}

\subsubsection[MET]{Model Evaluation Tools (MET)}

\subsubsection[EVS]{Ensemble Verification System (EVS)}


%TODO tabuľka softvéru

\section{Vizualizácia verifikácie}

\subsection{Scatterplot}

\subsection{Boxplot}

\subsection{Time series plot}